\documentclass[a4paper]{jarticle}
\usepackage{sice-si}
\usepackage{amsmath} 
\usepackage[dvipdfmx]{graphicx}


\begin{document}
%
% タイトルと著者名
\title{音声中に出現する特定キーワードの自動ゲイン調整を行う装置の開発} % 和文タイトル
\name{○佐々部 岳人,天野 俊一(流通経済大学)} % 著者名
\etitle{Development of a device that automatically adjusts the gain of specific keywords that appear in speech.} % 英文タイトル
\ename{○Gakuto Sasabe, and Shunichi Amano (Ryutsu Keizai University)}	%著者名(英)
%
% アブストラクト
\abst{
When we obtain information visually, it is possible to filter only the information we want to obtain, for example, by using a recommendation function for online shopping. On the other hand, in the auditory sense, technologies that uniformly cut noise in specific frequency bands in the outside world, such as noise cancellation, have been put to practical use, but there are still few technologies that cut specific information such as keywords that appear in speech. If such technology is put to practical use, it is expected to contribute to the improvement of productivity and creativity in work involving listening. In this study, we will develop a system that automatically adjusts the gain of speech corresponding to specific keywords. We will also examine the effect of this system on the user's task performance.
}
% タイトルの出力
\maketitle
%
% 本文
\section{緒言}
本稿では SICE SI 部門講演会 SI の予稿原稿を作成するための説明を行います.
SIでは予稿原稿としてPDFファイル形式のファイルを電子投稿していただくことを原則とさせていただいております.
ただし,電子化やネットワーク接続が困難な場合には個別に対応させていただきますので,プログラム委員会までご相談ください(Webサイトからお問い合わせできます).
%
\section{自動ゲイン調整装置}
\begin{figure}[htbp]
    \begin{center}
    \includegraphics[width=80mm]{system.PNG}
    \caption{The device and the configuration.}
    \label{fig:system}
    \end{center}
    \end{figure}

本稿で使用した装置の構成をFig.\ref{fig:system}に示す.
本装置はワイヤレスヘッドフォン,PC,マイクによって構成される.
PCには,Pythonで書かれたシステムが搭載されており,これによってマイクからの音に対して自動でゲイン調整を行う.
ワイヤレスヘッドフォンはノイズキャンセリング機能を有し,装置使用時はノイズキャンセリング機能を常にONの状態としている.
すなわちユーザは外界からの音をマイクによってのみ得ることとなる.
以下に,マイクから音を拾って,ユーザーがゲイン調整された音を聞くまでのシステム内の流れを示す.
まず,外界からの音をマイクによって拾い,Google社のSpeech recogntionによって音声のテキスト化が行われる.
次に,生成されたテキストは検閲ワード検索クラスに送られ,あらかじめ設定された検閲ワードがテキスト中に含まれていないかどうか検索が行われる.
もし,テキスト中に検閲ワードが含まれていた場合は,含まれていた検閲ワードと,検閲ワードを見つけたという情報がゲイン調整クラスに送られる.
一方で,外界からの音はPythonライブラリであるPyaudioによってチャンクごとの音声データに分けられ,ゲイン調整クラスに送られる.(Fig.\ref{fig:system}の上部分)
ゲイン調整クラスでは,同じくpythonライブラリであるPycawによって音声のゲイン調整が行われる.
このゲイン調整の度合いは前述した,発見した検閲ワードの種類によってあらかじめ設定することができる.(例えば,”こんにちは”というワードを発見したらゲインを0とする等)
最後にゲイン調整が行われた音声がワイヤレスヘッドフォンに送られ,ユーザーは音声を聞くことができる.

\section{ゲイン調整システムの検証実験}
\subsection{システムによる音のカットの確認}
\subsubsection{実験概要}
\subsubsection{実験結果}
\subsection{システムがユーザーのタスク遂行に与える影響}
\subsubsection{実験概要}
システムがユーザーのタスク遂行に与える影響を調べるため,4人の男女(男性:3人,女性:1人)に対してAlternative Uses Test(参考文献)を行った.
Alternative Uses Test(以下AUT)とは,被験者に日用品の新たな使い方のアイデアを思いつく限り解答させるタスクである.
例えばお題が「鉛筆」であれば通常用途として「メモを取る」等が考えられるが代替用途は,「黒板を示すのに使う」「箸の代わりとして使う」などである.
(創造性の評価指標を乗せる?)
さらに,今回は雑音環境下での(適切なワードか再考)システムの有効性を測るため,AUTの最中に実験者からアイデア出しに関するアドバイスを行った.
実験の手順を以下に示す
\begin{enumerate}
    \item 被験者が装置を装着する
    \item 実験者が被験者にAUTのやり方を説明する
    \item 実験者が被験者にお題を伝え,AUTを始める
    \item 随時,実験者から被験者に対してアイデア出しに関するアドバイスを伝える
    \item AUTを始めてから3分後,AUTを終了する
    \item (2)~(4)をもう一度繰り返す
\end{enumerate}
実験は対面で行い,実験者の対面に被験者が座った.実験の様子を図〇に示す.
被験者は,AUTのお題として1回目の試行では「ボールペン」,2回目の試行では「靴下」についてアイデア出しを行うように指示された.
実験中に出たアイデアは逐次,A4用紙に記入してもらった.
また,AUT実施中の実験者からのアドバイスはお題に関するもの(お題が靴下であれば「素材が布であることを考えると面白いアイデアが思いつくかもしれません.」)と,
お題によらないもの(「誰が使うかを考えてみるといいアイデアが思いつくかもしれません」等)をAUT開始から30秒毎に一言ずつ計5回行った.
実験を行うにあたりシステム強使用条件とシステム弱使用条件の2つの条件を設け,それぞれの条件につき2人ずつ実験に参加してもらった.
なお,被験者は自分がどちらの条件の被験者になったかは知らされていない.
システム強使用条件では,「はじめてください」というキーワードをゲインを0%(無音状態)にするトリガーとし,
「終わってください」というキーワードをゲインを100%(制限なし状態)とするトリガーにした.
すなわち,システム強使用条件では,具体的な実験に対する説明以外のAUTをおこなっている部分では無音となる.
また,システム弱使用条件では,ゲイン調整を行うトリガーを設けず,実験の全ての段階でゲインを100%(制限なし状態)とした.
実験中,被験者の斜め前に設置されたビデオカメラにより,各人の頭部,胸部運動の様子を映像として記録した.
記録したデータは各人の活動量の産出と評価のために使用された.
実験後,被験者には簡単な心理アンケートに答えてもらった.
アンケート結果は被験者のアイデア出しへの自己評価や,実験者に対する印象などを調査するために使用された.
\subsubsection{実験結果}
\section{ディスカッション}
%
%
%参考文献
\begin{thebibliography}{99}
\bibitem{SI}
	計測太郎,制御花子:
	``SICE SI予稿原稿の書き方(サンプル)'',  
   {\it 計測自動制御学会SI部門講演会SICE-SI予稿集}, 
    pp.0000--0000 (20??)
\end{thebibliography}
%
%
%
\end{document}

